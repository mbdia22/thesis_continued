\begin{abstract}
Reduced-form credit risk models have become the standard for pricing defaultable securities, yet the choice of recovery assumption—Recovery of Market Value (RMV) versus Recovery of Face Value (RFV)—is often driven by mathematical tractability rather than empirical realism. This paper extends the affine term structure framework to derive closed-form expressions for stochastic duration and convexity under the RFV specification, contrasting them with the widely used RMV baseline. We quantify the "model risk" inherent in this choice, demonstrating that the RMV assumption systematically overestimates interest rate sensitivity. Our numerical analysis reveals that RMV duration exceeds RFV duration by approximately 6\% when default intensity is positively correlated with interest rates. Furthermore, RMV significantly overstates convexity, implying that standard models overestimate the natural gamma protection of corporate bonds. These findings suggest that relying on RMV-based hedging ratios leads to systematic over-hedging of interest rate risk for distressed debt portfolios.
\end{abstract}
