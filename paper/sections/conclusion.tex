\section{Conclusion}

This paper has generalized the reduced-form credit risk framework to incorporate the Recovery of Face Value (RFV) assumption, providing a rigorous comparison with the standard Recovery of Market Value (RMV) baseline. By deriving semi-analytical solutions for stochastic duration and convexity under RFV, we have quantified the "model risk" associated with the choice of recovery specification.

Our analysis demonstrates that the RMV assumption is not a neutral mathematical convenience but a structural choice that systematically biases risk sensitivities upward. We found that RMV models overestimate duration by approximately 6\% and significantly overstate convexity. This bias stems from the implicit assumption in RMV that recovery values decline when interest rates rise (via the bond price channel), a mechanic that is absent in the fixed-recovery RFV framework.

For practitioners managing portfolios of distressed debt (where $\Lambda_1 > 0$ and default risk is material), relying on standard RMV-based risk systems will lead to systematic over-hedging. Specifically, hedging interest rate risk based on RMV duration would result in selling more treasury futures or buying more gamma protection than is necessary to offset the true economic risk of the position. As such, we advocate for the use of RFV-consistent pricing kernels when managing credit-sensitive assets where legal recovery is tied to par value.
