\section{Introduction}

The valuation of defaultable bonds requires a joint specification of the risk-free term structure, the default intensity, and the recovery mechanism. Since the seminal contributions of \cite{Jarrow1995} and \cite{Duffie1999}, reduced-form models have dominated the academic and practitioner landscape. Among these, the Recovery of Market Value (RMV) assumption has achieved ubiquity, primarily because it allows the defaultable bond to be priced as if it were a risk-free bond discounted at a risk-adjusted rate $R(t) = r(t) + L\lambda(t)$. While mathematically convenient, this assumption implies that recovery is a fraction of the pre-default market value—a circularity that is often at odds with legal bankruptcy proceedings where claims are typically based on the Face Value (Par) of the debt.

This paper challenges the convenience of the RMV assumption by rigorously analyzing the Recovery of Face Value (RFV) specification, where the recovery payment is a fixed fraction of the bond's principal. While the distinction may appear technical, we demonstrate that it has profound implications for the risk management of credit portfolios. The RMV assumption induces a "double sensitivity" to interest rates: as rates rise, the bond price falls, and consequently, the recovery value (being a fraction of that price) falls as well. In contrast, the RFV assumption provides a "buffer effect," as the recovery cash flow remains fixed regardless of the pre-default market value.

Building on the sensitivity analysis of \cite{Dial2019}, which characterized the stochastic duration of RMV bonds, this paper addresses a critical gap in the literature by explicitly deriving the sensitivity dynamics under RFV. We extend the affine term structure framework to incorporate the RFV mechanism, deriving semi-analytical solutions for both stochastic duration and convexity.

Our contributions are threefold. First, we provide a tractable implementation of the RFV pricing kernel within a Vasicek framework, utilizing the forward measure to handle the independence between the default time and the recovery payoff. Second, we present a comparative numerical analysis demonstrating that the RMV assumption consistently overestimates both duration and convexity. Specifically, we find that the hedging error—defined as the relative difference between RFV and RMV duration—approaches -6\% in environments where default intensity is positively correlated with interest rates. Finally, we show that the convexity bias is substantial, suggesting that standard RMV models exaggerate the convexity protection (gamma) embedded in corporate bonds.

The remainder of this paper is organized as follows. Section 2 outlines the theoretical framework and the mathematical derivation of the pricing equations. Section 3 details the numerical implementation and the calibration of the Vasicek parameters. Section 4 presents the results of the sensitivity analysis, quantifying the duration and convexity gaps. Section 5 concludes with implications for portfolio hedging.
